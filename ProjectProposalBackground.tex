
\section{Background}

We came across a document made in 2003 that outlined the possibility of implementing a medical information system through a “smart card”. The paper outlines how a “smart card” would implement QRF technology, that when scanned, would bring up the patients medical history. However, this method has the medical history actually being stored in the card itself. We want to design a database and use, lets say a health card, as just an ID, rather than the storage of the data itself.
Before developing a database, and the app to access the database, we first need to determine what relevant information we would need to be stored in the database. When talking to a paramedic, he gave this as a response to what information would be helpful:
Nursing homes carry a transfer document that (should) contain updated information on them such as: medications, allergies, and diagnosed disorders, including say if they had a stroke or have dementia, what level of cognition or communication is normal for them. This kind of information gives us as paramedics a tipping point for how we assess our patients or how we expect to communicate with them or have them do the same with us. That would be very helpful for anyone we meet. Even if it's only medical history and allergies, we can decipher a lot from that. Even recent visits to hospital gives us a pattern of problems with a patient. Currently the hospital can scan their card and get all prior hospital visits and diagnoses on discharges, but it's not available or compact enough for us to get. If somehow our laptop forms, which we have to manually input all information we can attain could get the aforementioned information it would largely influence our practice. Currently there are 2-3 different documenting systems in use in Ontario. So no unfortunately it's not standardized, that being said, neither is the hospital retrieval software, so an external device and software would help to standardize at least the ohip card aspect of retrieval.
We are creating the system based off the fact that we have acquired all the information, so we are focusing on the storing and displaying of the data, rather than the acquisition of it. Therefore, we need to decide how to effectively display and store the data. 

