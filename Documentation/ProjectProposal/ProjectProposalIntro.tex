
\section{Introduction}
\par Information is a vital resource, especially in the context of Healthcare. When a patient's health is at risk, their medical history becomes the most important piece of information in providing them with appropriate care \cite{web1}.

At present, there exists no secure, liftime record of your health history \cite{Street2014}. When faced with a problem, healthcare providers are left without the big picture, and often fumble for generic solutions, instead of catering care to each individual. In Ontario, the government is addressing this problem through a new initiative called eHealth Ontario; a group focused on researching how to best connect patients and providers through digital technology and information \cite{web1}. The initative has developed a framework for building the future of ``interoperable electronic health records (EHR) for Ontarians'', which outlines and defines how a comprehensive EHR may be developed \cite{b1}.

Among technical details, the blueprint defines how a information access portals may be used to distribute and present information to stakeholders, ensuring the most important and appropriate information is available when it is needed most. The blueprint further identifies key areas that may require access portals, such as hospitals, community care facilities and general practitioners \cite{Street2014}.

\iffalse where access portals may be

 in the most effective and streamlined manner. The blueprint identifies key areas,


centralized system with standardized records, accessed by different portals may be used to

 The group has developed a technical blueprint for how health information may

The blueprint is the framework for building ``interoperable electrnoic health records for Ontarions'' \fi

\iffalse The Ontario government is looking to solve this problem through the eHealth Ontario initiative, a group researching how best to impliment digital technologies that connect patients and providers through information \cite{web1}.\fi

One situation where this information gap is the most noticiable is when patients recieve care from Emergency First Responders (EFRs). Today, EFRs respond to emergencies knowing little or nothing about the people in their care. This is unforunate, considering how beneficial information is in improving patient outcomes. \iffalse For example, nursing homes often carry documents that contain health information on their residents. This information, including medications, allergies and diagonsed disorders gives EFRs an important base-knowledge for how assessing a patient. \fi

The objective of this capstone is to therefore develop an information management system that stores health records for access by patients, and their healthcare providers. Specifically, the goal of the capstone is to develop software that can be used by Emergency First Responders to access patient health records while responding to a call. This will provide EFRs with information vital to providing patients with the best care possible.

In addition to software, hardware will be developed that provides EFRs with a means to quickly identify patients and access there information on the system. This hardware will also include a real-time vitals tracker, which has the potential to track vitals (i.e. Heart Rate) deemed useful by EFRs.
