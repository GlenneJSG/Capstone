
\section{Introduction}

Information is a vital resource, espcially in the context of Healthcare. When a patient's health is at risk, their medical history becomes the most important piece of information in providing them with appropriate care.

At present, in Ontario, there exists no secure, liftime record of your health history. When faced with a problem, healthcare providers are left without the big picture, and often fumble for generic solutions, instead of catering care to each individual.

One situation where this information gap is the most noticiable is when patients recieve care from Emergency First Responders (EFRs). Today, EFRs respond to emergencies knowing little or nothing about the people in their care. This is unforunate, considering how beneficial information is in improving patient outcomes. \iffalse For example, nursing homes often carry documents that contain health information on their residents. This information, including medications, allergies and diagonsed disorders gives EFRs an important base-knowledge for how assessing a patient. \fi

The objective of this capstone is to therefore develop an information management system that stores health records for access by patients, and their healthcare providers. Specifically, the goal of the capstone is to develop software that can be used by Emergency First Responders to access patient health records while responding to a call. This will provide EFRs with information vital to providing patients with the best care possible. In addition to software, hardware will be developed that provides EFRs with a means to quickly identify patients and access there information on the system. This hardware will also include a real-time vitals tracker, which has the potential to track vitals (i.e. Heart Rate) deemed useful by EFRs.
