\documentclass{article}

\usepackage{booktabs}
\usepackage{tabularx}

\title{Title of Project}

\author{Team Name
		\\ Taylor de Vet
    \\ Glenne Grossman
		\\ Jamal Habash
		\\ Cameron Nowikow
}

\date{}



\begin{document}

\newpage

\maketitle

We write introduction stuff here

\section{Objectives}

To develop an information management system that stores health records for access by patients, and their healthcare providers. Specifically, the goal of the capstone is to develop software that can be used by Emergency First Responders (EFRs) to access patient health records while responding to a call. This will provide EFRs with information vital to providing patients with the best care possible. In addition to software, hardware will be developed that provides EFRs with a means to quickly identify patients and access there information on the system. This hardware will also include a real-time vitals tracker, which has the potential to track vitals (i.e. Heart Rate) deemed useful by EFRs.

\section{Resources}
The following is a list of resources that have been used in the research process thus far.
\begin{itemize}
\item Currently we are in communication with 2 Ontario Paramedics who have provided us insight on the various issues and difficulties that they run into on a daily basis.
\item We are also in communication with Dr. Aleksandar Jeremic and Dr. Hubert deBruin to gather their insight on this project.
\item We have access to a Bioinstrumentation laboratory at McMaster University which will allow us to test our hardware components both on patients and patient simulators to ensure everything is working properly and safely.
\end{itemize}



\section{Description}

\subsection{Motivation}
Cam
\subsection{Background}
All of us
\subsection{Project details milestone and completion}
Glenne
\section{Scheduling}
Glenne
\section{Assumptions and Risks}

\subsection{Assumptions}

As of present, it is assumed that there is no centralized health record that contains the cumulative health information of Ontario residents. The information required for this system is not available today, and we are developing software based on the assumption, that in the future, this information will need to be collected.

For this capstone, the software being built is focused on how health information is to be handeled, and the best way to present this information to the user, specifically EFRs. As a component of the capstone, we will brainstorm the best methods by which to collect and aggregate health records, however the means by which to collect these records will not be built.

\subsection{Risks}

Software development is inherently difficult to estimate and schedule. While we have done our best to estimate the time required to reach our goals, we do not have experience with large-scale software projects. Without experience, we do not truly know the length of time each component of the system will take to develop, and there is a risk we may not meet our inital development goals. To mitigate this risk, we will be having weekly meetings to gauge our progress, allowing us to modify our goals as required. Additonally, we have been conservative in our deliverable estimates, giving us ample opporunity to add, or remove features to the software depending on the availability of time.

\iffalse
have been conservative in our deliverable estimates, and will be having weekly meetings in order to guage the progress of development as we go along. This will allow us to remove, or add features to the software depending on the availability of time.




------------------

Random stuff :P ignore:

As of now, there is no centralized health record that contains the cumulative health information of Ontario residents.

We are developing software based on the assumption, that in the future, this information will need to be collected, and we will brainstorm the best methods by which to aggregate this information. The software being built is focused on how to handle the information and present the information, vs how the information is inputed into the system

It is assuming that we are not focused on how to input the information, so much as how to handle and present the information to the correct user.

It is assumed that the information required for this system is not available today

\fi

\section{Deliverables}
\subsection{Bronze}
The bronze category is the bare minimum of what is to be accomplished for the final product. The following are Bronze Level goals.
\begin{itemize}
  \item Working User Interface
	\begin{itemize}
	\item The user interface for this applivation is the most important aspect as this is what the Paramedic or Emergency personel will be looking at when inputting and acessing data. At this level, the user interface must have the following functionality. Firstly
	\item Firstly, the application must be compatible with iOS and Android devices. There is currently no standardization for the tools that Paramedics use and having an app that can be used on any platform makes it more readily useable with current technology.
	\item Secondly, this application must be able to access a database of health information assumed to exist and display said data on the screen of the users device. This will be done in a layout similar to the software
	\end{itemize}
  \item Another entry in the list
\end{itemize}
Taylor
\section{Summary}
Cam





\end{document}
