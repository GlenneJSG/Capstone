\documentclass{article}

\usepackage{booktabs}
\usepackage{tabularx}

\title{Title of Project}

\author{Team Name
		\\ Taylor de Vet
    		\\ Glenne Grossman
		\\ Jamal Habash
		\\ Cameron Nowikow
}

\date{}



\begin{document}

\newpage

\maketitle

We write introduction stuff here

\section{Objectives}
Jamal

\section{Resources}
Taylor


\section{Description}

\subsection{Motivation}
Cam
\subsection{Background}
All of us
\subsection{Project details milestone and completion}
Glenne
\section{Scheduling}
Glenne
\section{Assumptions and Risks}
Jamal
\section{Deliverables}

The goals for this project are ranked as Gold, Silver and Bronze. The main software and hardware components of this project each have their own hierarchy of goals.
\begin{itemize}
  \item Software: Working User Interface.
The user interface for this application is the most important aspect as this is what the Paramedic or Emergency personnel will be looking at when inputting and accessing data. 
	\begin{itemize}
		\item At the bronze level the application must be compatible with iOS and Android devices. There is currently no standardization for the tools that Paramedics use and having an app that can be used on any platform makes it more readily useable with current technology. Secondly, this application must be able to access a database of health information (assumed to exist) and display said data on the screen of the user�s device. This will be done in a layout similar to the software currently used by Ontario Paramedics. Lastly, the software must be able to read in information from the hardware component and display it on the user�s screen. 
		\item At the silver level, the application should be reading information in from the hardware and displaying it real time on the app. This can then be recorded back in the database to be used at a later date. Additionally, the application should have the ability to scan the barcode off of a health card to access the individual�s information rather than searching in the database. 
		\item At the gold level, the app should have the ability to send the data from the hardware to a hospital in real time so when the patient arrives, the doctors already have an idea of the history of the patient�s vitals during the traumatic event. Additionally, the paramedic should have the ability to update the database through the app if there is any pertinent information to the patient�s health from the current traumatic event.
	\end{itemize}
  \item Hardware: Vitals Tracker
	In conjunction with having health information from the database, a paramedic should have the ability to have information in real time from a vitals tracker in conjunction in order to make the most educated decisions about the patients.  
	\begin{itemize}
		\item At the bronze level, the vitals tracker will be able to track the heart rate of the user and will communicate with the application via Bluetooth. 
		\item At the silver level the vitals tracker will have the ability to track multiple vitals at once and send the information to the application. 
		\item At the gold level, each patient will be able to have their own vitals tracker that connects with the application and displays their own personal information. 
\end{itemize}




\section{Summary}
Cam





\end{document}
