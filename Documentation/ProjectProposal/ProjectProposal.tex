\documentclass{article}

\usepackage{booktabs}
\usepackage{tabularx}

\title{Title of Project}

\author{Team Name
		\\ Taylor de Vet
    \\ Glenne Grossman
		\\ Jamal Habash
		\\ Cameron Nowikow
}

\date{}



\begin{document}

\newpage

\maketitle

We write introduction stuff here

\section{Objectives}
Jamal

\section{Resources}
Taylor


\section{Description}

\subsection{Motivation}
Cam
\subsection{Background}
All of us
\subsection{Project details milestone and completion}
Glenne
\section{Scheduling}
Glenne
\section{Assumptions and Risks}
Jamal
\section{Deliverables}
\subsection{Bronze}
The bronze category is the bare minimum of what is to be accomplished for the final product. The following are Bronze Level goals. 
\begin{itemize}
  \item Working User Interface
	\begin{itemize}
	\item The user interface for this applivation is the most important aspect as this is what the Paramedic or Emergency personel will be looking at when inputting and acessing data. At this level, the user interface must have the following functionality. Firstly
	\item Firstly, the application must be compatible with iOS and Android devices. There is currently no standardization for the tools that Paramedics use and having an app that can be used on any platform makes it more readily useable with current technology.
	\item Secondly, this application must be able to access a database of health information assumed to exist and display said data on the screen of the users device. This will be done in a layout similar to the software 
	\end{itemize}
  \item Another entry in the list
\end{itemize}
Taylor
\section{Summary}
Cam





\end{document}
