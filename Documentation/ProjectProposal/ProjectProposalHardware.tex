\documentclass{article}

\usepackage{booktabs}
\usepackage{tabularx}

\title{Hardware Design Proposal}


\begin{document}

\newpage

\maketitle
\section{Hardware Design Component}

To adequately track the vitals of a patient, there are 2 hardware components required; the sensor, and the transmitter.
\subsection{Sensor}
As this is a preliminary design, only one vital will be tracked as proof of modality so the focus will be on heart rate and blood oxygen saturation. 
A simple heart rate sensor will be constructed consisting of an infrared and red wavelength emitter and receiver as well as an amplification circuit. 
The 2 LEDS are pulsed on the skin of the patient and the reflectance of both wavelengths is measured by detectors. IR and red light scatter differently 
depending on the amount of blood in the skin at that time and the detectors pick up the reflected light from both and output the corresponding current and 
voltage. This is then amplified and is sent to a microcontroller. \par

To decrease the size of the overall sensor, an ATmega microcontroller can be used with the appropriate capacitors, oscillators, and power source to avoid the 
use of a full Arduino board. This chip will receive signals in from the pulse oximetry circuit and can calculate the heart rate, blood oxygen concentration and
 will create a pulse wave to show heart activity. These signals can then be sent out through a Wi-Fi Transceiver module to the user’s mobile device. 
\subsection{Transmitter}
All modern communication devices, such as tablets and mobile phones, can connect to auxiliary devices via Bluetooth and Wi-Fi. Bluetooth communication has 
limitations on how many devices can be connected to a parent device at any given time with the ideal number being 4. \cite{apple1} As the primary use for this device
 will be for mass accidents, being limited to 4 patients per device is impractical. Wireless internet networks such as Wi-Fi are less limited and allow for 
 10 devices to be connected at a time when using a mobile hotspot source. \cite{} When a paramedic is out in the field, a mobile internet source will be provided
 from a mobile hotspot and will have the ability to have 10 devices actively communicating with it at a time. This also allows for information to be sent to 
 the hospital in real time whereas Bluetooth communication would be limited to when the patient has already reached the hospital. Because of this we have 
 chosen to connect to our sensors through Wi-Fi with the receiver being the devices own internal Wi-Fi receiver. 
To transmit the data from the pulse oximetry sensor to the mobile device, a serial Wi-Fi transceiver module will be used. With both devices connected to 
the same wireless network, data will able to be sent serially from the sensor to the host device.  


\end{document}
