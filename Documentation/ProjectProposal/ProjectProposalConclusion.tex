\section{Conclusion}

Throughout the development process, both the software and hardware will be tested to ensure it meets design requirements.

In terms of software, each module will be verified using test cases that have known outcomes to ensure functionality. For example, for the patient information retrieval, we will use several different pre-built, mock electronic medical records to test and verify how the program accesses and displays information. A successful test is one that displays all the information accurately, with no missing pieces. Mock electronics records will be formulated in a way that tests all cases for information retrieval, including any edge cases. \iffalse For software, the majority of test cases can be determined by using the functional requirements to explore how each component of the system interacts and testing all possibilities.
The hardware will be tested in a similar way, using test cases to verify the hardware meets specification. \fi

In regards to hardware, the device will be tested in a number of conditions to ensure it remains operational and meets device specifications. For example, in the case of the pulse sensor circuit, the device will be tested to ensure robustness, and that it meets functional requirements on people with a number of different skin types and sizes. Additionally, the hardware can be tested and compared against a certified medical device (i.e. Fitbit), to ensure the pulse is being sense correctly.

The ability to transfer data from the sensor to a mobile device can be tested through a simplistic mobile application that will display the information on the users screen.
