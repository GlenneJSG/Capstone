
\section{Software Design Component}

The software system we are designing will function as a portal to Electronic Health Records in Ontario. The main user of the system will be paramedics. In the future, we hope users will extend to all levels of the medical profession as well as the general public.
We will design an application targeted at paramedics that works on both iOS and Android phones.This application will allow paramedics to view patient information that they do not normally have access too. The app will have a loading screen, a login page, a home page, patient pages accessed by scanning health cards or inputting the health card number.


\subsection{Technology}
In order to create the Android and iOS app we will require certain technologies.
Firstly, the app will be written using React Native a javascript framework that allows apps to be written for Android and iOS concurrently. The app will require a database to store the patient information. For this, we will use Node.js however, we are assuming that in the future if our app is put to use we will easily adapt to how the EHR are stored by Ontario.


\subsection{Functional Design}
The app will have basic functional requirements as well as features we would like to implement once the basic requirements are completed.
The basic functional requirements are that the device follows the standards outlined on the ehealth blueprint documents put out by eHealth Ontario.
This document guides us through the appropriate exchange and access of information within the app.
Functionally, it will need to have a secure log-in screen for paramedics. A home screen with buttons leading to history, settings, option to connect to WiFi hardware device, and the health card scanner.
Once a hardware device or a health card are connected it will then need to bring up a patient page and show the sensor data and/or the patients EHR. The history screen for the basic requirements will simply show what patients the paramedic has treated but will not allow the paramedic more access to their information once the paramedic has closed the patient's EHR.
In future versions the paramedic will be able to record information about the particular visit into the EHR, and will be able to view these additions on the history screen. In the future additions, the sensed data will also be stored and be visible in the EHR so medical personell at a hospital or later at a doctor's visit will be able to view the patients vitals at the time of the paramedics visit.


\subsection{Non-Functional Design}
The app must have secure information storage and transmission by the standards outlined in the eHealth Blueprint. The app must have a cohesive look, including size and colour scheme in order to be easy to navigate. The app must be available in english and french. The app will be made so that the average paramedic is able to use it (assuming paramedics are males and females adults with above secondary education) and will use terms that are familiar to them.
The app must be able to run on Android and iOS devices that still recieve updates from their providers. The app must not have harsh colouring or flashing lights in order to prevent harm to the paramedics.
